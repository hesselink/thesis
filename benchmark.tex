\chapter{Benchmarking multirec}
\cite{benchmark}

\begin{itemize}
\item Describe what the benchmark is
\item Describe what Alexey has already done
\item Describe what I don't agree with and what my view is
\item Describe what my extension has improved
\end{itemize}

For the past years, research on generic programming in Haskell has
been a hot topic, and as a result, the number of libraries for
generics programming in Haskell has exploded. On the one hand, this
is a good thing: it allows us to compare and evaluate different
approaches to generic programming. On the other hand, this makes it
hard to choose a library for generic programming, especially since it
is difficult to compare the different libraries.

Alexey Rodriguez has developed a benchmark for generic programming
libraries \cite{benchmark} that aims to alleviate this problem. It
selects a range of types of problems that generic programming can
solve, and rates a range of libraries on if they are able to solve
these problems. Additionally, some other quality aspects like ease of
use are also judged. Using the results of this benchmark, it is
possible to select the right generic programming library for a certain
application.

Although the technical report does not include a benchmark for the
multirec \cite{multirec} library, this is included in Alexey
Rodriguez' thesis \cite{thesis_alexey}. The results of this benchmark
are reproduced in figure \Todo{add figure and ref}.
